%===========================================================
% Bachelor Project Report (Software Engineering + Databases)
% LaTeX Template (single-file)
%===========================================================
\documentclass[12pt,a4paper]{report}

% -----------------------
% Packages
% -----------------------
\usepackage[a4paper,margin=2.5cm]{geometry}
\usepackage[T1]{fontenc}
\usepackage[utf8]{inputenc} % If using modern engines (xelatex/lualatex), you can remove this.
\usepackage{lmodern}

\usepackage{graphicx}
\usepackage{float}
\usepackage{booktabs}
\usepackage{longtable}
\usepackage{array}
\usepackage{enumitem}
\usepackage{setspace}
\usepackage{caption}
\usepackage{subcaption}
\usepackage{hyperref}

\usepackage{acronym}

% Bibliography (recommended)
\usepackage[
  backend=biber,
  style=ieee,
  sorting=nyt
]{biblatex}
\addbibresource{references.bib} % Create references.bib (see example at end)

% -----------------------
% Document metadata
% -----------------------
\hypersetup{
  pdftitle={Project Title},
  pdfauthor={Author Name(s)},
  colorlinks=true,
  linkcolor=blue,
  urlcolor=blue,
  citecolor=blue
}

\setlength{\parskip}{0.4em}
\setlength{\parindent}{0pt}
\onehalfspacing

% -----------------------
% Title info (edit)
% -----------------------
\newcommand{\ProjectTitle}{Project Title}
\newcommand{\ProjectSubtitle}{Software Engineering and Databases}
\newcommand{\AuthorNames}{Author Name(s)}
\newcommand{\Institution}{Institution Name}
\newcommand{\Course}{Bachelor Degree -- Course Name}
\newcommand{\Supervisor}{Supervisor: Name}
\newcommand{\SubmissionDate}{December 2025}

%===========================================================
\begin{document}

% -----------------------
% Title page
% -----------------------
\begin{titlepage}
  \centering
  \vspace*{2cm}

  {\Large \Institution \par}
  \vspace{0.5cm}
  {\large \Course \par}
  \vspace{2cm}

  {\Huge \bfseries \ProjectTitle \par}
  \vspace{0.4cm}
  {\Large \ProjectSubtitle \par}

  \vfill

  {\Large \AuthorNames \par}
  \vspace{0.5cm}
  {\large \Supervisor \par}
  \vspace{1cm}
  {\large \SubmissionDate \par}

  \vspace*{1cm}
\end{titlepage}

% -----------------------
% Abstract + Keywords
% -----------------------
\begin{abstract}
% 150--250 words.
% Include: problem/context, objectives, approach/method, key results, conclusion.
This report presents \textit{[project name]}, a system designed to \textit{[short purpose]}.
The main objective is to \textit{[objective 1]} while ensuring \textit{[objective 2]}.
The solution is based on \textit{[brief architecture]} and a relational database model that enforces
integrity through constraints and normalized design.
Evaluation includes functional validation and database-oriented testing, complemented by
a basic performance analysis of the critical queries.
Results show \textit{[main outcome]}, with limitations related to \textit{[key limitation]}.
\end{abstract}

\noindent\textbf{Keywords:} databases, software engineering, data modelling, relational design, testing

% -----------------------
% Optional: lists
% -----------------------
\tableofcontents
\listoffigures
\listoftables

%===========================================================
\chapter{Introduction}
% 1--3 pages typically.
\section{Context and Motivation}
% Why the problem matters; who cares; what is the domain.

\section{Problem Statement}
% Define the problem in 3--6 sentences. Avoid vague statements.

\section{Objectives}
% Bullet list is fine.
\begin{itemize}[leftmargin=1.2em]
  \item Objective 1: \textit{[...]}
  \item Objective 2: \textit{[...]}
  \item Objective 3: \textit{[...]}
\end{itemize}

\section{Scope and Limitations}
% Clarify boundaries: what is included/excluded.

\section{Document Structure}
% 1 short paragraph describing chapters.

%===========================================================
\chapter{Background and Related Work}
\section{Core Concepts}
% Example topics: ER modelling, normalization, ACID, indexes, REST, auth.
% Cite sources for definitions/claims.

\section{Related Systems / Approaches}
% Mention 2--8 references (bachelor-level is usually enough).
% Example: \cite{date2003introduction}

\section{Positioning of This Work}
% How your solution differs or what it focuses on.

%===========================================================
\chapter{Requirements Analysis}
\section{Stakeholders and Users}
% Who uses it and in what context.

\section{Functional Requirements}
% Numbered list helps traceability.
\begin{enumerate}[label=FR\arabic*, leftmargin=1.6em]
  \item \textbf{[Feature name]}: \textit{[requirement description]}
  \item \textbf{[Feature name]}: \textit{[requirement description]}
  \item \textbf{[Feature name]}: \textit{[requirement description]}
\end{enumerate}

\section{Non-Functional Requirements}
\begin{enumerate}[label=NFR\arabic*, leftmargin=1.8em]
  \item \textbf{Security}: \textit{[e.g., role-based access control for ...]}
  \item \textbf{Performance}: \textit{[e.g., critical search queries under X ms for N rows]}
  \item \textbf{Usability}: \textit{[e.g., key workflow in <= K steps]}
  \item \textbf{Maintainability}: \textit{[e.g., modular structure, documentation, tests]}
\end{enumerate}

\section{Use Cases / User Stories}
% A compact table is often appreciated.
\begin{table}[H]
\centering
\caption{Representative Use Cases}
\begin{tabular}{p{2.2cm} p{10.5cm}}
\toprule
\textbf{UC ID} & \textbf{Description} \\
\midrule
UC1 & User logs in and accesses the dashboard. \\
UC2 & User creates a new \textit{[entity]} and the system validates required fields. \\
UC3 & User searches \textit{[entity]} using filters and sorting. \\
\bottomrule
\end{tabular}
\end{table}

\section{Acceptance Criteria (Optional)}
% Concrete "done" statements linked to FR/NFR.

%===========================================================
\chapter{System Design}
\section{Architecture Overview}
% Include a diagram (component / layered architecture).
\begin{figure}[H]
  \centering
  \fbox{\parbox{0.85\textwidth}{\centering
    Placeholder: Architecture diagram (components and interactions)}}
  \caption{High-level architecture of the system.}
  \label{fig:architecture}
\end{figure}

\section{Main Components and Responsibilities}
% Describe modules/services and their responsibilities.

\section{Key Interactions (Optional)}
% Sequence diagram for critical workflow (create order, booking, etc.).

\section{Design Decisions and Trade-offs}
% Explain important choices (e.g., why relational DB, why indexing strategy).

%===========================================================
\chapter{Database Design}
\section{Conceptual Model (ER Diagram)}
\begin{figure}[H]
  \centering
  \fbox{\parbox{0.85\textwidth}{\centering
    Placeholder: ER diagram (entities, relationships, cardinalities)}}
  \caption{Entity-Relationship diagram.}
  \label{fig:erd}
\end{figure}

\section{Logical Model (Relational Schema)}
% Provide a clear schema description: tables, attributes, PK/FK, constraints.
% A table helps, but you can also describe it in text.
\begin{table}[H]
\centering
\caption{Example Schema Summary (adapt to your project)}
\begin{tabular}{p{3.2cm} p{9.5cm}}
\toprule
\textbf{Table} & \textbf{Notes (PK/FK/Constraints)} \\
\midrule
users & PK(user\_id), UNIQUE(email), NOT NULL(password\_hash) \\
orders & PK(order\_id), FK(user\_id $\rightarrow$ users), CHECK(status in ...) \\
order\_items & PK(order\_item\_id), FK(order\_id), FK(product\_id), CHECK(quantity > 0) \\
\bottomrule
\end{tabular}
\end{table}

\section{Normalization Discussion}
% State target normal form (e.g., 3NF) and justify.
% Mention any deliberate denormalization and why.

\section{Constraints and Integrity}
% Mention: NOT NULL, UNIQUE, CHECK, FK actions (ON DELETE/UPDATE).

\section{Physical Design (Indexes and Performance Rationale)}
% Show the critical queries and indexes.
% Optionally add a small EXPLAIN summary.

%===========================================================
\chapter{Implementation Overview}
% Not code listings; focus on mapping requirements to solution.
\section{Technology Stack}
% What you used + short justification.

\section{Feature Mapping (Requirements to Modules)}
% A traceability table is strong.
\begin{table}[H]
\centering
\caption{Traceability Example}
\begin{tabular}{p{2.2cm} p{7.0cm} p{3.5cm}}
\toprule
\textbf{Req} & \textbf{Implementation} & \textbf{Evidence} \\
\midrule
FR1 & \textit{[module/functionality]} & \textit{[screenshot/test]} \\
FR2 & \textit{[module/functionality]} & \textit{[screenshot/test]} \\
NFR1 & \textit{[security mechanism]} & \textit{[test/cfg]} \\
\bottomrule
\end{tabular}
\end{table}

\section{Security Considerations}
% Auth model, permissions, validation, parameterized queries, etc.

%===========================================================
\chapter{Testing and Evaluation}
\section{Test Strategy}
% Unit vs integration vs system; what tools.

\section{Test Results}
\begin{table}[H]
\centering
\caption{Testing Summary}
\begin{tabular}{p{4.5cm} p{6.5cm} p{2.5cm}}
\toprule
\textbf{Area} & \textbf{What was tested} & \textbf{Result} \\
\midrule
Business logic & \textit{[rule validations, edge cases]} & Pass \\
Database constraints & \textit{[FK integrity, CHECK constraints]} & Pass \\
API / UI flows & \textit{[UC1, UC2, UC3]} & Pass \\
\bottomrule
\end{tabular}
\end{table}

\section{Performance Evaluation (Basic but Real)}
% Example: timings for key queries; impact of indexes; dataset size.

%===========================================================
\chapter{Results and Discussion}
\section{Key Results}
% Tie outcomes to objectives and requirements.

\section{Limitations}
% Honest limitations: incomplete features, dataset limits, missing scalability, etc.

\section{Trade-offs}
% Explain choices and their consequences.

%===========================================================
\chapter{Conclusion and Future Work}
\section{Conclusion}
% Short summary: what you built, what you proved, main outcomes.

\section{Future Work}
\begin{itemize}[leftmargin=1.2em]
  \item \textit{[feature improvement]}
  \item \textit{[performance improvement]}
  \item \textit{[security improvement]}
\end{itemize}

%===========================================================
\printbibliography

%===========================================================
\appendix
\chapter{Appendix A: Full Schema Listing}
% Paste full schema (or include it).
% \verbatiminput{db/schema.sql} % if you use \usepackage{verbatim}

\chapter{Appendix B: Representative Queries}
% Include 3--10 important queries and short explanations.

\chapter{Appendix C: Additional Test Evidence}
% Logs, screenshots references, coverage summary, etc.

\end{document}

%===========================================================
% Example references.bib (create as a separate file)
%===========================================================
% @book{date2003introduction,
%   title     = {An Introduction to Database Systems},
%   author    = {Date, C. J.},
%   year      = {2003},
%   publisher = {Addison-Wesley}
% }
%
% @book{pressman2014software,
%   title     = {Software Engineering: A Practitioner’s Approach},
%   author    = {Pressman, Roger S. and Maxim, Bruce R.},
%   year      = {2014},
%   publisher = {McGraw-Hill}
% }
