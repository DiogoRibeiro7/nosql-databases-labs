%===========================================================
% Standard Bachelor Report Template (Software Engineering + Databases)
% Generic template for students to follow (not tied to a specific project)
%===========================================================
\documentclass[12pt,a4paper]{report}

\usepackage[a4paper,margin=2.5cm]{geometry}
\usepackage[T1]{fontenc}
\usepackage[utf8]{inputenc}
\usepackage{lmodern}

\usepackage{graphicx}
\usepackage{float}
\usepackage{booktabs}
\usepackage{array}
\usepackage{enumitem}
\usepackage{setspace}
\usepackage{hyperref}
\usepackage{caption}
\usepackage{acronym}
\usepackage{verbatim}

\usepackage[
  backend=biber,
  style=ieee,
  sorting=nyt
]{biblatex}
\addbibresource{references.bib}

\hypersetup{
  pdftitle={Bachelor Report Template},
  pdfauthor={Student Name(s)},
  colorlinks=true,
  linkcolor=blue,
  urlcolor=blue,
  citecolor=blue
}

\setlength{\parskip}{0.4em}
\setlength{\parindent}{0pt}
\onehalfspacing

% -----------------------
% Fields students must fill
% -----------------------
\newcommand{\ReportTitle}{Title of the Work}
\newcommand{\StudentNames}{Student Name(s)}
\newcommand{\Institution}{Institution Name}
\newcommand{\Course}{Course / Unit}
\newcommand{\Supervisor}{Supervisor / Instructor}
\newcommand{\SubmissionDate}{Month Year}

\begin{document}

%===========================================================
% Title page
%===========================================================
\begin{titlepage}
  \centering
  \vspace*{2cm}

  {\Large \Institution \par}
  \vspace{0.4cm}
  {\large \Course \par}

  \vspace{2cm}
  {\Huge \bfseries \ReportTitle \par}
  \vspace{1cm}

  {\Large \StudentNames \par}
  \vspace{0.5cm}
  {\large Supervisor: \Supervisor \par}

  \vfill
  {\large \SubmissionDate \par}
  \vspace*{1cm}
\end{titlepage}

%===========================================================
% Abstract
%===========================================================
\begin{abstract}
% 150--250 words.
% Write: (1) context/problem, (2) objectives, (3) approach, (4) evaluation, (5) results, (6) conclusion.
This report addresses \textit{[context/problem]} by proposing \textit{[solution/system]}.
The objectives are \textit{[objective 1]}, \textit{[objective 2]}, and \textit{[objective 3]}.
The proposed solution is based on \textit{[architecture/approach]} and a database model designed to enforce integrity
through \textit{[constraints/normalization]}.
Evaluation was performed using \textit{[testing approach]} and \textit{[performance/validation method]}.
Results show \textit{[main result]}, while the main limitations are \textit{[limitations]}.
\end{abstract}

\noindent\textbf{Keywords:} software engineering, databases, data modelling, constraints, testing

\tableofcontents
\listoffigures
\listoftables

%===========================================================
\chapter{Introduction}
\section{Context and Motivation}
Explain the real context. Describe why the problem exists and why it matters.
Give one or two concrete examples of failures or inefficiencies that motivate a solution.

\section{Problem Statement}
State the problem clearly in 3--6 sentences.
Avoid broad statements such as ``there is a need for better systems''.
Instead, define what is missing and what must be guaranteed.

\section{Objectives}
List 3--6 objectives. Each objective should be measurable.
\begin{itemize}[leftmargin=1.2em]
  \item \textbf{O1:} \textit{[objective]}
  \item \textbf{O2:} \textit{[objective]}
  \item \textbf{O3:} \textit{[objective]}
\end{itemize}

\section{Scope and Limitations}
Clarify what is included and excluded. Mention scale assumptions (number of users, records, etc.) if relevant.

\section{Document Structure}
Explain briefly what each chapter contains.

%===========================================================
\chapter{Background and Related Work}
\section{Background Concepts}
Summarize the concepts needed to understand your work.
For database projects, typical concepts include ER modelling, normalization, constraints, indexes, and transactions.
Cite sources for definitions and claims.

\section{Related Work}
Describe similar systems or approaches (academic or practical).
Explain what you reuse and what is different in your work.
Use 2--10 references depending on the assignment scope.

\section{Positioning}
One paragraph explaining the focus of this work (e.g., correctness and integrity vs feature richness).

%===========================================================
\chapter{Requirements Analysis}
\section{Users and Stakeholders}
Who will use the system? In what situations? What are their goals?

\section{Functional Requirements}
Write requirements as ``The system shall \dots'' statements.
Number them to enable traceability later.
\begin{enumerate}[label=FR\arabic*, leftmargin=1.6em]
  \item The system shall \textit{[do something]}.
  \item The system shall \textit{[do something]}.
  \item The system shall \textit{[do something]}.
\end{enumerate}

\section{Non-Functional Requirements}
Include at least 4 categories: integrity, performance, security, maintainability/usability.
\begin{enumerate}[label=NFR\arabic*, leftmargin=1.8em]
  \item \textbf{Integrity:} \textit{[constraints/invariants that must hold]}.
  \item \textbf{Performance:} \textit{[response time for key operations, dataset scale]}.
  \item \textbf{Security:} \textit{[who can do what; protection against common risks]}.
  \item \textbf{Maintainability:} \textit{[modularity, tests, documentation]}.
\end{enumerate}

\section{Use Cases / User Stories}
List 4--8 key use cases. Describe the main success path and the main failure path.
\begin{table}[H]
\centering
\caption{Use Cases}
\begin{tabular}{p{2.2cm} p{10.5cm}}
\toprule
\textbf{UC ID} & \textbf{Description} \\
\midrule
UC1 & \textit{[short description of workflow]} \\
UC2 & \textit{[short description of workflow]} \\
UC3 & \textit{[short description of workflow]} \\
\bottomrule
\end{tabular}
\end{table}

\section{Acceptance Criteria (Recommended)}
Write 1--2 acceptance criteria for each key requirement (what must be true to consider it done).

%===========================================================
\chapter{System Design}
\section{Architecture Overview}
Describe the architecture at a high level.
Include a diagram (layered architecture is acceptable for bachelor projects).

\begin{figure}[H]
  \centering
  \fbox{\parbox{0.85\textwidth}{\centering
    Placeholder: Architecture diagram (components and interactions)}}
  \caption{High-level architecture.}
  \label{fig:architecture}
\end{figure}

\section{Component Responsibilities}
Describe each component/module and what it owns.
Keep it short and precise.

\section{Key Workflows (Optional)}
For 1--2 critical workflows, provide a sequence diagram or a structured description.

\section{Design Decisions}
Explain 3--6 important decisions and trade-offs.
Examples: relational vs NoSQL, normalization level, transaction approach, indexing strategy.

%===========================================================
\chapter{Database Design}
\section{Conceptual Model (ER Diagram)}
Provide an ER diagram with cardinalities.
Explain the main entities and relationships.

\begin{figure}[H]
  \centering
  \fbox{\parbox{0.85\textwidth}{\centering
    Placeholder: ER diagram with entities and cardinalities}}
  \caption{Entity-Relationship diagram.}
  \label{fig:erd}
\end{figure}

\section{Logical Model (Relational Schema)}
Describe tables, attributes, primary keys, foreign keys, and constraints.
Add a short schema summary table.

\begin{table}[H]
\centering
\caption{Schema Summary (template)}
\begin{tabular}{p{3.2cm} p{9.5cm}}
\toprule
\textbf{Table} & \textbf{Notes (PK/FK/Constraints)} \\
\midrule
\textit{table\_1} & PK(\textit{id}), FK(\textit{...}), UNIQUE(\textit{...}), CHECK(\textit{...}) \\
\textit{table\_2} & PK(\textit{id}), FK(\textit{...}), NOT NULL(\textit{...}) \\
\textit{table\_3} & PK(\textit{id}), FK(\textit{...}), CHECK(\textit{...}) \\
\bottomrule
\end{tabular}
\end{table}

\section{Normalization and Rationale}
State the target normal form (e.g., 3NF).
Justify the choice in 1--2 paragraphs.
Mention any denormalization and why it is acceptable.

\section{Integrity Constraints}
List the key invariants and how they are enforced:
PK/FK, uniqueness, domain checks, and referential actions (ON DELETE/ON UPDATE).

\section{Indexes and Query Rationale}
Provide 3--6 critical queries (in text) and explain which indexes support them.
Optionally include \texttt{EXPLAIN} summaries.

%===========================================================
\chapter{Implementation Overview}
\section{Technology Stack}
List the tools and technologies used and provide short justifications.

\section{Traceability (Requirements to Implementation)}
Provide a table connecting requirements to modules/endpoints/queries/tests.

\begin{table}[H]
\centering
\caption{Traceability Table (template)}
\begin{tabular}{p{2.2cm} p{7.0cm} p{3.5cm}}
\toprule
\textbf{Req} & \textbf{Implementation} & \textbf{Evidence} \\
\midrule
FR1 & \textit{[module / endpoint / workflow]} & \textit{[test / screenshot]} \\
FR2 & \textit{[module / endpoint / workflow]} & \textit{[test / screenshot]} \\
NFR1 & \textit{[constraints / security control]} & \textit{[schema / test]} \\
\bottomrule
\end{tabular}
\end{table}

\section{Security and Validation}
Summarize authentication/authorization and input validation.
Mention parameterized queries and error handling conventions.

%===========================================================
\chapter{Testing and Evaluation}
\section{Testing Strategy}
Describe the levels of testing (unit, integration, system).
Specify what is tested at each level.

\section{Test Results}
Summarize tests in a table and mention the most important edge cases.

\begin{table}[H]
\centering
\caption{Testing Summary (template)}
\begin{tabular}{p{4.5cm} p{6.5cm} p{2.5cm}}
\toprule
\textbf{Area} & \textbf{What was tested} & \textbf{Result} \\
\midrule
Business rules & \textit{[edge cases and invalid states]} & \textit{Pass/Fail} \\
Database integrity & \textit{[FK, CHECK, UNIQUE]} & \textit{Pass/Fail} \\
Critical workflows & \textit{[UC1..UCk]} & \textit{Pass/Fail} \\
\bottomrule
\end{tabular}
\end{table}

\section{Performance Evaluation (Basic but Real)}
State dataset size, test method, and results.
Include at least one critical query and show the impact of indexes if possible.

%===========================================================
\chapter{Results and Discussion}
\section{Results}
Summarize what was achieved, mapped to objectives and requirements.

\section{Limitations}
List honest limitations: missing features, scale limits, incomplete security, etc.

\section{Trade-offs}
Explain important trade-offs (correctness vs complexity, normalization vs query speed, etc.).

%===========================================================
\chapter{Conclusion and Future Work}
\section{Conclusion}
One short section summarizing the outcome and main evidence (tests, constraints, evaluation).

\section{Future Work}
List realistic improvements.
\begin{itemize}[leftmargin=1.2em]
  \item \textit{[improvement 1]}
  \item \textit{[improvement 2]}
  \item \textit{[improvement 3]}
\end{itemize}

\printbibliography

%===========================================================
\appendix
\chapter{Appendix A: SQL Schema (excerpt)}
\begin{verbatim}
-- Put an excerpt of your schema here, not the whole thing if it is very long.
-- Include CREATE TABLE + key constraints + key indexes.
\end{verbatim}

\chapter{Appendix B: Representative Queries}
\begin{verbatim}
-- Include 3--10 queries that demonstrate your system.
-- Show at least: one join query, one aggregation query, and one integrity-related query.
\end{verbatim}

\chapter{Appendix C: Extra Evidence}
% Screenshots references, test logs, coverage output, etc.

\end{document}

%===========================================================
% Minimal references.bib example (separate file)
%===========================================================
% @book{date2003introduction,
%   title     = {An Introduction to Database Systems},
%   author    = {Date, C. J.},
%   year      = {2003},
%   publisher = {Addison-Wesley}
% }
%
% @book{pressman2014software,
%   title     = {Software Engineering: A Practitioner’s Approach},
%   author    = {Pressman, Roger S. and Maxim, Bruce R.},
%   year      = {2014},
%   publisher = {McGraw-Hill}
% }
